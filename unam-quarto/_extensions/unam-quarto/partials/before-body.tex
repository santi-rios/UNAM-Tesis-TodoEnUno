
% \frontmatter % Use roman page numbering style (i, ii, iii, iv...) for the pre-content pages

\pagestyle{plain} % Default to the plain heading style until the tesis style is called for the body content

%----------------------------------------------------------------------------------------
%	PORTADA
%----------------------------------------------------------------------------------------

% Primera portada
\begin{titlepage}
    \begin{center}
        \vspace*{-1.5cm} % Ajusta este valor conforme sea necesario
        
        % Logo de la UNAM
        \includegraphics[width=3.5cm]{$if(tesis.logo)$ $tesis.logo$ $endif$}\\[0.5cm]
        
        % Nombre de la Universidad
        {\large \textbf{UNIVERSIDAD NACIONAL AUTÓNOMA DE MÉXICO}}\\[0.4cm]
        
        % Programa de posgrado
        {\large \textbf{\MakeUppercase{$tesis.programa$}}}\\[0.3cm]
        
        % Facultad y área
        {\large \MakeUppercase{$tesis.facultad$}}\\[0.2cm]
        {\large \MakeUppercase{$tesis.departmento$}}\\[1cm]
        
        % Título de la tesis
        {\Large \textbf{$title$}}\\[1cm]
        
        % Texto "Tesis"
        {\LARGE \textbf{T E S I S}}\\[0.5cm]
        
        % Texto de grado
        {\large QUE PARA OPTAR POR EL GRADO DE:}\\[0.3cm]
        {\Large \textbf{$tesis.grado$}}\\[1cm]
        
        % Autor
        {\large PRESENTA:}\\[0.3cm]
        {\LARGE \textbf{$tesis.nombre$}}\\[1.0cm]
        
        % Tutor y comité
        {\small \textbf{TUTOR PRINCIPAL DE TESIS:}}\\
        {\small $if(tesis.supervisor.name)$ $tesis.supervisor.name$ $endif$}\\
        {\small $if(tesis.supervisor.faculty)$ $tesis.supervisor.faculty$ $endif$}\\[0.5cm]
        
        {\small \textbf{COMITÉ TUTOR:}}\\
        {\small $if(tesis.sinodales.uno)$ $tesis.sinodales.uno$ $endif$}\\
        {\small $if(tesis.sinodales.unofacultad)$ $tesis.sinodales.unofacultad$ $endif$}\\[0.3cm]
        {\small $if(tesis.sinodales.dos)$ $tesis.sinodales.dos$ $endif$}\\
        {\small $if(tesis.sinodales.dosfacultad)$ $tesis.sinodales.dosfacultad$ $endif$}\\[1cm]
        
        % Lugar y fecha
        % \vfill
        % $if(book.date)$
        % {\small \textbf{{$tesis.lugar$} \hfill {$book.date$}}}\\
        % $endif$
        
    \end{center}
\end{titlepage}


% Primera página de \missingfigure
\newpage
\begin{center}
    \missingfigure[figcolor=white]{\MakeUppercase{\LARGE{Página reservada para Derechos de Autor por parte de la Dirección de Bibliotecas.}}}
\end{center}

% Tercera página de \missingfigure
\newpage
\begin{center}
    \missingfigure{\MakeUppercase{\LARGE{Página reservada para Oficio de Jurado.}}}
\end{center}


\newpage
% \begin{agradecimientos}
    \begin{center}
        {\large \textbf{Agradecimientos Institucionales}}\\[0.3cm]
        \includegraphics[width=3.5cm]{$if(tesis.logoB)$ $tesis.logoB$ $endif$}\\[0.5cm]
        Este trabajo de Maestría se realizó gracias al Posgrado en Ciencias Biológicas de la Universidad Nacional Autónoma de México (UNAM) y al financiamiento otorgado por el Consejo Nacional de Ciencia y Tecnología (CONACyT) a través de la beca al CVU 1191018 y al proyecto PAPIIT IA205723.

        Deseo expresar mi más sincero agradecimiento a la Universidad Nacional Autónoma de México por haberme recibido nuevamente como parte de su comunidad estudiantil. 
        
        Quiero extender mi agradecimiento a la Facultad de Ciencias de la UNAM por brindar las facilidades y al Laboratorio de Biología Animal Experimental por proporcionar las instalaciones, el equipo y las cámaras. Agradezco al director de esta tesis, el Dr. Alonso, por sus ayudas.
        
        Quiero expresar mi más sincero agradecimiento a todos los miembros del jurado y, en especial, al comité tutoral, el Dr. Jean Pascal Morin y la Dra. María de la Luz Navarro, por dedicar su tiempo a leer esta tesis y proporcionar valiosos comentarios que enriquecieron este trabajo. Su apoyo constante y sus observaciones constructivas fueron fundamentales para lograr un mejor resultado. Gracias por su compromiso y generosidad a lo largo de este proceso.
        
        Por último, gracias a los investigadores y estudiantes que ayudaron a darle forma a este trabajo: gracias a la Dra. Angélica Zepeda y su equipo de trabajo por enseñarme y guiarme con los temas de neurogénesis. Gracias a la Dra. Julieta Rosell García y a los Maestros Diego Ángeles Valdez y Jalil Rasgado Toledo por ser quienes ayudaron con toda la parte de estadística. Gracias los doctores Dan Chitwood y Sarah Percival por incluirme en sus proyectos, donde aprendí muchísimo sobre programación y estadística. Por último, gracias las psicólogas Amanda Sánchez Pérez y Jimena Arroyo Pérez por ayudarme con la parte de trastornos, comportamiento, experimentos y ser quienes discutían y revisaban el proyecto conmigo. Sin ustedes no hubiera sido posible realizar este trabajo.
    \end{center}
% \end{agradecimientos}




\newpage
% \begin{agradecimientos}
    \begin{center}
        {\large \textbf{Agradecimientos Personales}}\\[0.3cm]
        Lleno de sinapsis agotadas y neuronas al borde del colapso, es momento de agradecer a quienes hicieron posible que esta tesis sobre vea la luz.

        A mi familia, que aún no entiende del todo qué es lo que hago, pero siempre me apoyó incondicionalmente. 

    \end{center}

\newpage
% \begin{agradecimientos}
    \begin{center}
        Para Cami y Reni por ser mis compañeras de vida y de tesis. Por ser mi apoyo incondicional, mi motivación y mi inspiración. Por ser mis críticas más duras y mis mayores fans. Por ser mis mejores amigas y mis compañeras de tesis. Por ser mis hermanas y mis cómplices. Por ser mis confidentes y mis consejeras. Por ser mis compañeras de piso y mis compañeras de tesis. Por ser mis compañeras de via
    \end{center}

\newpage
\begin{center}
    \missingfigure[figcolor=white]{\MakeUppercase{\LARGE{Página en blanco.}}}
\end{center}

% \renewcommand*\contentsname{Tabla de contenidos}
\newpage
% \tableofcontents

$if(tesis.agradecimientos)$
\input{"$tesis.agradecimientos$"}
$endif$

\newpage
\tableofcontents