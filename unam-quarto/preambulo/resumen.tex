{\LARGE \textbf{\textsc{Resumen}}}\\[0.5cm]

En este trabajo investigamos el impacto del narcisismo encubierto en la práctica científica. Este tipo de narcisismo se manifiesta a menudo como una aparente obsesión por la perfección, aunque en realidad suele traducirse en un sabotaje pasivo-agresivo del método científico. Analizaremos cómo el narcisismo encubierto, disfrazado de falsa modestia, socava los pilares fundamentales de la ciencia: el pensamiento flexible, la crítica constructiva y la humildad intelectual. Después de todo, la frase "Soy muy brillante, pero el mundo está contra mí" suele significar "Soy insufrible con los demás". 