Este trabajo de Maestría se realizó gracias al Posgrado en Ciencias Biológicas de la Universidad Nacional Autónoma de México (UNAM) y al financiamiento otorgado por el Consejo Nacional de Ciencia y Tecnología (CONACyT) a través de la beca al CVU 1191018 y al proyecto PAPIIT IA205723.

Deseo expresar mi más sincero agradecimiento a la Universidad Nacional Autónoma de México por haberme recibido nuevamente como parte de su comunidad estudiantil. 

Quiero extender mi agradecimiento a la Facultad de Ciencias de la UNAM por brindar las facilidades y al Laboratorio de Biología Animal Experimental por proporcionar las instalaciones, el equipo y las cámaras. Agradezco al director de esta tesis, el Dr. Alonso, por sus ayudas.

Quiero expresar mi más sincero agradecimiento a todos los miembros del jurado y, en especial, al comité tutoral, el Dr. Jean Pascal Morin y la Dra. María de la Luz Navarro, por dedicar su tiempo a leer esta tesis y proporcionar valiosos comentarios que enriquecieron este trabajo. Su apoyo constante y sus observaciones constructivas fueron fundamentales para lograr un mejor resultado. Gracias por su compromiso y generosidad a lo largo de este proceso.

Por último, gracias a los investigadores y estudiantes que ayudaron a darle forma a este trabajo: gracias a la Dra. Angélica Zepeda y su equipo de trabajo por enseñarme y guiarme con los temas de neurogénesis. Gracias a la Dra. Julieta Rosell García y a los Maestros Diego Ángeles Valdez y Jalil Rasgado Toledo por ser quienes ayudaron con toda la parte de estadística. Gracias los doctores Dan Chitwood y Sarah Percival por incluirme en sus proyectos, donde aprendí muchísimo sobre programación y estadística. Por último, gracias las psicólogas Amanda Sánchez Pérez y Jimena Arroyo Pérez por ayudarme con la parte de trastornos, comportamiento, experimentos y ser quienes discutían y revisaban el proyecto conmigo. Sin ustedes no hubiera sido posible realizar este trabajo.