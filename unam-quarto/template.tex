% Options for packages loaded elsewhere
\PassOptionsToPackage{unicode}{hyperref}
\PassOptionsToPackage{hyphens}{url}
\PassOptionsToPackage{dvipsnames,svgnames,x11names}{xcolor}
%
\documentclass[
]{article}

\usepackage{amsmath,amssymb}
\usepackage{iftex}
\ifPDFTeX
  \usepackage[T1]{fontenc}
  \usepackage[utf8]{inputenc}
  \usepackage{textcomp} % provide euro and other symbols
\else % if luatex or xetex
  \usepackage{unicode-math}
  \defaultfontfeatures{Scale=MatchLowercase}
  \defaultfontfeatures[\rmfamily]{Ligatures=TeX,Scale=1}
\fi
\usepackage{lmodern}
\ifPDFTeX\else  
    % xetex/luatex font selection
\fi
% Use upquote if available, for straight quotes in verbatim environments
\IfFileExists{upquote.sty}{\usepackage{upquote}}{}
\IfFileExists{microtype.sty}{% use microtype if available
  \usepackage[]{microtype}
  \UseMicrotypeSet[protrusion]{basicmath} % disable protrusion for tt fonts
}{}
\makeatletter
\@ifundefined{KOMAClassName}{% if non-KOMA class
  \IfFileExists{parskip.sty}{%
    \usepackage{parskip}
  }{% else
    \setlength{\parindent}{0pt}
    \setlength{\parskip}{6pt plus 2pt minus 1pt}}
}{% if KOMA class
  \KOMAoptions{parskip=half}}
\makeatother
\usepackage{xcolor}
\setlength{\emergencystretch}{3em} % prevent overfull lines
\setcounter{secnumdepth}{-\maxdimen} % remove section numbering
% Make \paragraph and \subparagraph free-standing
\ifx\paragraph\undefined\else
  \let\oldparagraph\paragraph
  \renewcommand{\paragraph}[1]{\oldparagraph{#1}\mbox{}}
\fi
\ifx\subparagraph\undefined\else
  \let\oldsubparagraph\subparagraph
  \renewcommand{\subparagraph}[1]{\oldsubparagraph{#1}\mbox{}}
\fi

\usepackage{color}
\usepackage{fancyvrb}
\newcommand{\VerbBar}{|}
\newcommand{\VERB}{\Verb[commandchars=\\\{\}]}
\DefineVerbatimEnvironment{Highlighting}{Verbatim}{commandchars=\\\{\}}
% Add ',fontsize=\small' for more characters per line
\usepackage{framed}
\definecolor{shadecolor}{RGB}{241,243,245}
\newenvironment{Shaded}{\begin{snugshade}}{\end{snugshade}}
\newcommand{\AlertTok}[1]{\textcolor[rgb]{0.68,0.00,0.00}{#1}}
\newcommand{\AnnotationTok}[1]{\textcolor[rgb]{0.37,0.37,0.37}{#1}}
\newcommand{\AttributeTok}[1]{\textcolor[rgb]{0.40,0.45,0.13}{#1}}
\newcommand{\BaseNTok}[1]{\textcolor[rgb]{0.68,0.00,0.00}{#1}}
\newcommand{\BuiltInTok}[1]{\textcolor[rgb]{0.00,0.23,0.31}{#1}}
\newcommand{\CharTok}[1]{\textcolor[rgb]{0.13,0.47,0.30}{#1}}
\newcommand{\CommentTok}[1]{\textcolor[rgb]{0.37,0.37,0.37}{#1}}
\newcommand{\CommentVarTok}[1]{\textcolor[rgb]{0.37,0.37,0.37}{\textit{#1}}}
\newcommand{\ConstantTok}[1]{\textcolor[rgb]{0.56,0.35,0.01}{#1}}
\newcommand{\ControlFlowTok}[1]{\textcolor[rgb]{0.00,0.23,0.31}{#1}}
\newcommand{\DataTypeTok}[1]{\textcolor[rgb]{0.68,0.00,0.00}{#1}}
\newcommand{\DecValTok}[1]{\textcolor[rgb]{0.68,0.00,0.00}{#1}}
\newcommand{\DocumentationTok}[1]{\textcolor[rgb]{0.37,0.37,0.37}{\textit{#1}}}
\newcommand{\ErrorTok}[1]{\textcolor[rgb]{0.68,0.00,0.00}{#1}}
\newcommand{\ExtensionTok}[1]{\textcolor[rgb]{0.00,0.23,0.31}{#1}}
\newcommand{\FloatTok}[1]{\textcolor[rgb]{0.68,0.00,0.00}{#1}}
\newcommand{\FunctionTok}[1]{\textcolor[rgb]{0.28,0.35,0.67}{#1}}
\newcommand{\ImportTok}[1]{\textcolor[rgb]{0.00,0.46,0.62}{#1}}
\newcommand{\InformationTok}[1]{\textcolor[rgb]{0.37,0.37,0.37}{#1}}
\newcommand{\KeywordTok}[1]{\textcolor[rgb]{0.00,0.23,0.31}{#1}}
\newcommand{\NormalTok}[1]{\textcolor[rgb]{0.00,0.23,0.31}{#1}}
\newcommand{\OperatorTok}[1]{\textcolor[rgb]{0.37,0.37,0.37}{#1}}
\newcommand{\OtherTok}[1]{\textcolor[rgb]{0.00,0.23,0.31}{#1}}
\newcommand{\PreprocessorTok}[1]{\textcolor[rgb]{0.68,0.00,0.00}{#1}}
\newcommand{\RegionMarkerTok}[1]{\textcolor[rgb]{0.00,0.23,0.31}{#1}}
\newcommand{\SpecialCharTok}[1]{\textcolor[rgb]{0.37,0.37,0.37}{#1}}
\newcommand{\SpecialStringTok}[1]{\textcolor[rgb]{0.13,0.47,0.30}{#1}}
\newcommand{\StringTok}[1]{\textcolor[rgb]{0.13,0.47,0.30}{#1}}
\newcommand{\VariableTok}[1]{\textcolor[rgb]{0.07,0.07,0.07}{#1}}
\newcommand{\VerbatimStringTok}[1]{\textcolor[rgb]{0.13,0.47,0.30}{#1}}
\newcommand{\WarningTok}[1]{\textcolor[rgb]{0.37,0.37,0.37}{\textit{#1}}}

\providecommand{\tightlist}{%
  \setlength{\itemsep}{0pt}\setlength{\parskip}{0pt}}\usepackage{longtable,booktabs,array}
\usepackage{calc} % for calculating minipage widths
% Correct order of tables after \paragraph or \subparagraph
\usepackage{etoolbox}
\makeatletter
\patchcmd\longtable{\par}{\if@noskipsec\mbox{}\fi\par}{}{}
\makeatother
% Allow footnotes in longtable head/foot
\IfFileExists{footnotehyper.sty}{\usepackage{footnotehyper}}{\usepackage{footnote}}
\makesavenoteenv{longtable}
\usepackage{graphicx}
\makeatletter
\def\maxwidth{\ifdim\Gin@nat@width>\linewidth\linewidth\else\Gin@nat@width\fi}
\def\maxheight{\ifdim\Gin@nat@height>\textheight\textheight\else\Gin@nat@height\fi}
\makeatother
% Scale images if necessary, so that they will not overflow the page
% margins by default, and it is still possible to overwrite the defaults
% using explicit options in \includegraphics[width, height, ...]{}
\setkeys{Gin}{width=\maxwidth,height=\maxheight,keepaspectratio}
% Set default figure placement to htbp
\makeatletter
\def\fps@figure{htbp}
\makeatother
% definitions for citeproc citations
\NewDocumentCommand\citeproctext{}{}
\NewDocumentCommand\citeproc{mm}{%
  \begingroup\def\citeproctext{#2}\cite{#1}\endgroup}
\makeatletter
 % allow citations to break across lines
 \let\@cite@ofmt\@firstofone
 % avoid brackets around text for \cite:
 \def\@biblabel#1{}
 \def\@cite#1#2{{#1\if@tempswa , #2\fi}}
\makeatother
\newlength{\cslhangindent}
\setlength{\cslhangindent}{1.5em}
\newlength{\csllabelwidth}
\setlength{\csllabelwidth}{3em}
\newenvironment{CSLReferences}[2] % #1 hanging-indent, #2 entry-spacing
 {\begin{list}{}{%
  \setlength{\itemindent}{0pt}
  \setlength{\leftmargin}{0pt}
  \setlength{\parsep}{0pt}
  % turn on hanging indent if param 1 is 1
  \ifodd #1
   \setlength{\leftmargin}{\cslhangindent}
   \setlength{\itemindent}{-1\cslhangindent}
  \fi
  % set entry spacing
  \setlength{\itemsep}{#2\baselineskip}}}
 {\end{list}}
\usepackage{calc}
\newcommand{\CSLBlock}[1]{\hfill\break\parbox[t]{\linewidth}{\strut\ignorespaces#1\strut}}
\newcommand{\CSLLeftMargin}[1]{\parbox[t]{\csllabelwidth}{\strut#1\strut}}
\newcommand{\CSLRightInline}[1]{\parbox[t]{\linewidth - \csllabelwidth}{\strut#1\strut}}
\newcommand{\CSLIndent}[1]{\hspace{\cslhangindent}#1}

% \usepackage[utf8]{fontenc}
\usepackage[T1]{fontenc}
\usepackage[autostyle=true]{csquotes}
\usepackage[colorinlistoftodos, textsize=tiny]{todonotes} % insertar comentarios y notas
\usepackage[spanish]{babel}
% \usepackage[disable]{todonotes} # descomentar si quieres QUITAR notas y comentarios

% \usepackage{acro}
% \ac \Ac acronym first time
% \acs \Acs short form
% \acl \Acl longform

% \acsetup{
% 	make-links = true 
% 	}

% \DeclareAcronym{5-ht}{
%     short = 5-HT,
%     long = Serotonina
% }

% \DeclareAcronym{5-htp}{
%     short = 5-HTP,
%     long = 5-Hidroxitriptófano
% }


%----------------------------------------------------------------------------------------
%	MARGINS
%----------------------------------------------------------------------------------------

% Márgenes para impresión
% \geometry{
% 	headheight=4ex,
% 	includehead,
% 	includefoot
% }

% Márgenes para web
% \geometry{
%     left=2.5cm,
%     right=2.5cm,
%     top=2.5cm,
%     bottom=2.5cm
% }


% \raggedbottom

% \AtBeginDocument{
% \hypersetup{pdftitle=\ttitle} % Set the PDF's title to your title
% \hypersetup{pdfauthor=\authorname} % Set the PDF's author to your name
% \hypersetup{pdfkeywords=\keywordnames} % Set the PDF's keywords to your keywords
% }
\makeatletter
\@ifpackageloaded{caption}{}{\usepackage{caption}}
\AtBeginDocument{%
\ifdefined\contentsname
  \renewcommand*\contentsname{Table of contents}
\else
  \newcommand\contentsname{Table of contents}
\fi
\ifdefined\listfigurename
  \renewcommand*\listfigurename{List of Figures}
\else
  \newcommand\listfigurename{List of Figures}
\fi
\ifdefined\listtablename
  \renewcommand*\listtablename{List of Tables}
\else
  \newcommand\listtablename{List of Tables}
\fi
\ifdefined\figurename
  \renewcommand*\figurename{Figure}
\else
  \newcommand\figurename{Figure}
\fi
\ifdefined\tablename
  \renewcommand*\tablename{Table}
\else
  \newcommand\tablename{Table}
\fi
}
\@ifpackageloaded{float}{}{\usepackage{float}}
\floatstyle{ruled}
\@ifundefined{c@chapter}{\newfloat{codelisting}{h}{lop}}{\newfloat{codelisting}{h}{lop}[chapter]}
\floatname{codelisting}{Listing}
\newcommand*\listoflistings{\listof{codelisting}{List of Listings}}
\makeatother
\makeatletter
\makeatother
\makeatletter
\@ifpackageloaded{caption}{}{\usepackage{caption}}
\@ifpackageloaded{subcaption}{}{\usepackage{subcaption}}
\makeatother
\ifLuaTeX
  \usepackage{selnolig}  % disable illegal ligatures
\fi
\usepackage{bookmark}

\IfFileExists{xurl.sty}{\usepackage{xurl}}{} % add URL line breaks if available
\urlstyle{same} % disable monospaced font for URLs
\hypersetup{
  pdftitle={El Narcisista Encubierto en el Laboratorio Cuando el Soy Solo un Perfeccionista Sabotea el Método Científico},
  pdfauthor={string},
  pdfkeywords={template, demo},
  colorlinks=true,
  linkcolor={blue},
  filecolor={Maroon},
  citecolor={Blue},
  urlcolor={Blue},
  pdfcreator={LaTeX via pandoc}}

\title{El Narcisista Encubierto en el Laboratorio Cuando el Soy Solo un
Perfeccionista Sabotea el Método Científico}

% \programa{Posgrado en Ciencias Biológicas}

% % \facultad{Facultad de Ciencias} % Nombre de la facultad
% 
% +
% \departmento{Biología Experimental} % Nombre del departamento
% 

% % \grado{MAESTRO EN CIENCIAS BIOLÓGICAS}
% 
% % \supervisor{Dr.~Narcissus}
% 
% % \supervisorfac{Thespiae in Boeotia}
% 
% 
% 
% 
% 


% % \author{string}
% 


% 
% % \university{}
% 


% % \group{}
% 

% \setcounter{tocdepth}{3} % The depth to which the document sections are printed to the table of contents

% %   \author{string}{%
%   % 
\begin{document}

% \frontmatter % Use roman page numbering style (i, ii, iii, iv...) for the pre-content pages

\pagestyle{plain} % Default to the plain heading style until the tesis style is called for the body content

%----------------------------------------------------------------------------------------
%	PORTADA
%----------------------------------------------------------------------------------------

% Primera portada
\begin{titlepage}
    \begin{center}
        \vspace*{-1.5cm} % Ajusta este valor conforme sea necesario
        
        % Logo de la UNAM
        \includegraphics[width=3.5cm]{ figuras/unam.png }\\[0.5cm]
        
        % Nombre de la Universidad
        {\large \textbf{UNIVERSIDAD NACIONAL AUTÓNOMA DE MÉXICO}}\\[0.4cm]
        
        % Programa de posgrado
        {\large \textbf{\MakeUppercase{Posgrado en Ciencias
Biológicas}}}\\[0.3cm]
        
        % Facultad y área
        {\large \MakeUppercase{Facultad de Ciencias}}\\[0.2cm]
        {\large \MakeUppercase{Biología Experimental}}\\[1cm]
        
        % Título de la tesis
        {\Large \textbf{El Narcisista Encubierto en el Laboratorio
Cuando el Soy Solo un Perfeccionista Sabotea el Método
Científico}}\\[1cm]
        
        % Texto "Tesis"
        {\LARGE \textbf{T E S I S}}\\[0.5cm]
        
        % Texto de grado
        {\large QUE PARA OPTAR POR EL GRADO DE:}\\[0.3cm]
        {\Large \textbf{MAESTRO EN CIENCIAS BIOLÓGICAS}}\\[1cm]
        
        % Autor
        {\large PRESENTA:}\\[0.3cm]
        {\LARGE \textbf{Santiago García-Rios}}\\[1.0cm]
        
        % Tutor y comité
        {\small \textbf{TUTOR PRINCIPAL DE TESIS:}}\\
        {\small  Dr.~Narcissus }\\
        {\small  Thespiae in Boeotia }\\[0.5cm]
        
        {\small \textbf{COMITÉ TUTOR:}}\\
        {\small  Dr.~Socrates }\\
        {\small  Academy of Athens }\\[0.3cm]
        {\small  Dr.~Plato }\\
        {\small  Academy of Athens }\\[1cm]
        
        % Lugar y fecha
        % \vfill
        %         
    \end{center}
\end{titlepage}


% Primera página de \missingfigure
\newpage
\begin{center}
    \missingfigure[figcolor=white]{\MakeUppercase{\LARGE{Página reservada para Derechos de Autor por parte de la Dirección de Bibliotecas.}}}
\end{center}

% Tercera página de \missingfigure
\newpage
\begin{center}
    \missingfigure{\MakeUppercase{\LARGE{Página reservada para Oficio de Jurado.}}}
\end{center}


\newpage
% \begin{agradecimientos}
    \begin{center}
        {\large \textbf{Agradecimientos Institucionales}}\\[0.3cm]
        \includegraphics[width=3.5cm]{ figuras/unamposgrado.png }\\[0.5cm]
        Este trabajo de Maestría se realizó gracias al Posgrado en Ciencias Biológicas de la Universidad Nacional Autónoma de México (UNAM) y al financiamiento otorgado por el Consejo Nacional de Ciencia y Tecnología (CONACyT) a través de la beca al CVU 1191018 y al proyecto PAPIIT IA205723.

        Deseo expresar mi más sincero agradecimiento a la Universidad Nacional Autónoma de México por haberme recibido nuevamente como parte de su comunidad estudiantil. 
        
        Quiero extender mi agradecimiento a la Facultad de Ciencias de la UNAM por brindar las facilidades y al Laboratorio de Biología Animal Experimental por proporcionar las instalaciones, el equipo y las cámaras. Agradezco al director de esta tesis, el Dr. Alonso, por sus ayudas.
        
        Quiero expresar mi más sincero agradecimiento a todos los miembros del jurado y, en especial, al comité tutoral, el Dr. Jean Pascal Morin y la Dra. María de la Luz Navarro, por dedicar su tiempo a leer esta tesis y proporcionar valiosos comentarios que enriquecieron este trabajo. Su apoyo constante y sus observaciones constructivas fueron fundamentales para lograr un mejor resultado. Gracias por su compromiso y generosidad a lo largo de este proceso.
        
        Por último, gracias a los investigadores y estudiantes que ayudaron a darle forma a este trabajo: gracias a la Dra. Angélica Zepeda y su equipo de trabajo por enseñarme y guiarme con los temas de neurogénesis. Gracias a la Dra. Julieta Rosell García y a los Maestros Diego Ángeles Valdez y Jalil Rasgado Toledo por ser quienes ayudaron con toda la parte de estadística. Gracias los doctores Dan Chitwood y Sarah Percival por incluirme en sus proyectos, donde aprendí muchísimo sobre programación y estadística. Por último, gracias las psicólogas Amanda Sánchez Pérez y Jimena Arroyo Pérez por ayudarme con la parte de trastornos, comportamiento, experimentos y ser quienes discutían y revisaban el proyecto conmigo. Sin ustedes no hubiera sido posible realizar este trabajo.
    \end{center}
% \end{agradecimientos}




\newpage
% \begin{agradecimientos}
    \begin{center}
        {\large \textbf{Agradecimientos Personales}}\\[0.3cm]
        Lleno de sinapsis agotadas y neuronas al borde del colapso, es momento de agradecer a quienes hicieron posible que esta tesis sobre vea la luz.

        A mi familia, que aún no entiende del todo qué es lo que hago, pero siempre me apoyó incondicionalmente. 

    \end{center}

\newpage
% \begin{agradecimientos}
    \begin{center}
        Para Cami y Reni por ser mis compañeras de vida y de tesis. Por ser mi apoyo incondicional, mi motivación y mi inspiración. Por ser mis críticas más duras y mis mayores fans. Por ser mis mejores amigas y mis compañeras de tesis. Por ser mis hermanas y mis cómplices. Por ser mis confidentes y mis consejeras. Por ser mis compañeras de piso y mis compañeras de tesis. Por ser mis compañeras de via
    \end{center}

\newpage
\begin{center}
    \missingfigure[figcolor=white]{\MakeUppercase{\LARGE{Página en blanco.}}}
\end{center}

% \renewcommand*\contentsname{Tabla de contenidos}
\newpage
% \tableofcontents

\input{"preambulo/agradecimientos\_institucionales.tex"}

\newpage
\tableofcontents
\renewcommand*\contentsname{Table of contents}
{
\hypersetup{linkcolor=}
\setcounter{tocdepth}{3}
\tableofcontents
}
\subsection{Introduction}\label{sec-intro}

\begin{itemize}
\tightlist
\item[$\square$]
  Comenzar con una oración de tópico
  (\href{https://en.wikipedia.org/wiki/Inverted_pyramid_(journalism)}{pirámide
  invertida})
\item[$\square$]
  Revisión concisa de literatura pertinente
\end{itemize}

\href{https://writingcenter.gmu.edu/writing-resources/imrad/imrad-reports-introductions}{Ver
más}

¿Qué pasa cuando mezclas a alguien que piensa que es Ramón y Cajal con
la necesidad de probar que su hipótesis favorita es un la única
correcta? \textbf{Spoiler}: caos epistemológico. Exploraremos cómo el
\textbf{\emph{narcisismo encubierto}}, disfrazado de falsa modestia,
corroe los pilares de la ciencia: pensamiento flexible, crítica
constructiva y humildad intelectual. Porque, admitámoslo, la frase
``\emph{Soy muy brillante, pero el mundo está contra mi}'' suele
significar ``\emph{Soy insufrible con los demás}''.

Imagina a esa persona que publica ``Ugh, odio ser tan inteligente en un
mundo de idiotas'' en facebook mientras bebe un vino de oro y escucha
``mejor música que la tuya''. Bienvenido al \textbf{\emph{narcisismo
encubierto}}: el arte de disfrazar la autoimportancia bajo un manto de
falsa modestia. Aquí exploraremos cómo el cerebro de estos individuos
convierte la autocrítica en un \emph{humblebrag} (autoelogio disfrazado
de queja) y por qué sus neuronas parecen organizar un festival de cine
en su honor.

\subsubsection{Ejemplos de Narcisismo
Encubierto}\label{ejemplos-de-narcisismo-encubierto}

En general, el narcisismo encubierto se caracteriza por una combinación
de victimización, autoimagen grandiosa y sabotaje pasivo-agresivo
(Table~\ref{tbl-actitudes}).

\begin{longtable}[]{@{}
  >{\centering\arraybackslash}p{(\columnwidth - 2\tabcolsep) * \real{0.3933}}
  >{\centering\arraybackslash}p{(\columnwidth - 2\tabcolsep) * \real{0.6067}}@{}}
\caption{Actitudes y Consecuencias}\label{tbl-actitudes}\tabularnewline
\toprule\noalign{}
\begin{minipage}[b]{\linewidth}\centering
Actitud Narcisista Encubierta
\end{minipage} & \begin{minipage}[b]{\linewidth}\centering
Consecuencias en Ciencia/Medios
\end{minipage} \\
\midrule\noalign{}
\endfirsthead
\toprule\noalign{}
\begin{minipage}[b]{\linewidth}\centering
Actitud Narcisista Encubierta
\end{minipage} & \begin{minipage}[b]{\linewidth}\centering
Consecuencias en Ciencia/Medios
\end{minipage} \\
\midrule\noalign{}
\endhead
\bottomrule\noalign{}
\endlastfoot
``Sufro por ser tan bueno en esto'' & Rechazo a metodologías rigurosas
(ej: Theranos). \\
``Las críticas son envidia'' & Cámaras de eco que perpetúan errores. \\
Autoimagen de mártir & Sabotaje de colaboraciones científicas. \\
Manipulación pasivo-agresiva & Desinformación pública (ej: Goop, teorías
sin base). \\
\end{longtable}

A continuación, algunos ejemplos de narcisismo encubierto:

\textbf{\emph{Chuck McGill (Better Call Saul)}}
(Figure~\ref{fig-chuck}):

Actitudes narcisistas encubiertas:

\begin{quote}
``Soy el mártir de la ética'': Se presenta como guardián de la ley, pero
en realidad desprecia a Jimmy por amenazar su autoimagen de ``hombre
superior''. Victimización crónica: ``Mi electrosensibilidad me hace
sufrir'' (traducción: Necesito controlar cada situación). Sabotaje
pasivo-agresivo: Usa tácticas legales ``correctas'' para destruir a
Jimmy, mientras se cree moralmente intachable.
\end{quote}

Consecuencias:

\begin{quote}
Destruye su relación familiar. Pierde credibilidad al revelarse su
hipocresía. Su rigidez mental lo lleva al aislamiento y deterioro
mental.
\end{quote}

\begin{figure}

\centering{

\includegraphics[width=1.6in,height=\textheight]{figuras/chuck.png}

}

\caption{\label{fig-chuck}Chuck McGill fingiendo una enfermedad para
obtener validación de otras personas.}

\end{figure}%

\textbf{\emph{Elon Musk}}:

Actitudes:

\begin{quote}
``El mundo no está listo para mi genio'': Desestima críticas a sus
proyectos (ej: Hyperloop, Twitter/X) como ``envidia''.
\end{quote}

\begin{quote}
Victimización pública: ``Me atacan por pensar diferente'' (cuando
expertos señalan fallos técnicos).
\end{quote}

\textbf{\emph{Elizabeth Holmes (Theranos)}}:

Actitudes:

\begin{quote}
Victimización tecnológica: ``La industria me odia por innovar'' (cuando
ocultaba fallos técnicos).
\end{quote}

\begin{quote}
Autoimagen de ``elegida'': Voz impostada y narrativa de underdog para
encubrir fraude.
\end{quote}

Consecuencias:

\begin{quote}
Colapso de Theranos y desconfianza en startups de salud.
\end{quote}

\begin{quote}
Condena por fraude (su incapacidad de aceptar críticas la llevó a
ignorar evidencias).
\end{quote}

\subsubsection{Tables}\label{tables}

\begin{Shaded}
\begin{Highlighting}[]
\InformationTok{\textasciigrave{}\textasciigrave{}\textasciigrave{}\{r\}}
\InformationTok{\#| label: tbl{-}cars}
\InformationTok{\#| tbl{-}cap: This is my caption.}
\InformationTok{knitr::kable(head(mtcars))}
\InformationTok{\textasciigrave{}\textasciigrave{}\textasciigrave{}}
\end{Highlighting}
\end{Shaded}

The \texttt{\#\textbar{}} is what sets up our cross-references and you
can then reference the table as \texttt{@tbl-cars}. Note in order for
table numbering to work in Quarto, you \textbf{must} label your tables
with the \texttt{tbl-} prefix.

\subsection{Planteamiento del
problema}\label{planteamiento-del-problema}

\begin{quote}
¿Cómo afecta el narcisismo encubierto a la práctica científica? Estos
individuos suelen:
\end{quote}

\begin{itemize}
\tightlist
\item
  Descartar críticas como ``envidia académica''.
\item
  Adulterar datos para que coincidan con su ``brillante intuición''.
\item
  Desacreditar colegas con frases como ``Su metodología es\ldots{}
  interesante'' (traducción: ``Odio que tenga razón'').
\item
  ¿Es posible hacer ciencia rigurosa cuando tu cerebro interpreta
  ``revisión por pares'' como ``ataque personal''?
\end{itemize}

\begin{quote}
¿Cómo detectar a un narcisista encubierto antes de que te invite a su
monólogo sobre lo ``difícil que es ser tan inteligente y sensible''?
\end{quote}

La neurociencia sugiere que sus cerebros podrían tener áreas
hiperactivas en la corteza prefrontal (responsable de la autoevaluación)
y una amígdala que grita ``¡Mírame!'' cada vez que alguien recibe más
likes. ¿Es esto evolución o un bug cerebral?

\subsection{Justificación}\label{justificaciuxf3n}

Estudiar esto es urgente porque:

\begin{itemize}
\tightlist
\item
  La ciencia no es un monólogo, pero el narcisista encubierto la trata
  como su TED Talk personal.
\item
  El sesgo de confirmación se vuelve dogma cuando el investigador cree
  que ``mi teoría es tan obvia que hasta mi perro la entiende''.
\item
  La reputación de la ciencia se daña cuando los papers incluyen
  ``Agradezco a nadie, porque yo lo hice todo'' en la sección de
  acknowledgments.
\item
  Evitar que la palabra ``humildad'' sea secuestrada por narcisistas.
\item
  Diseñar apps que detecten mensajes pasivo-agresivos en redes sociales
  (``NeuroFilter: ¿Seguro que quieres publicar eso?'').
\item
  Salvar a la humanidad de conversaciones que empiezan con ``Soy
  demasiado empático, es mi maldición''
\end{itemize}

\subsection{Antecedentes}\label{antecedentes}

\begin{itemize}
\tightlist
\item[$\square$]
  Incluir suficiente literatura para entender la razón y contexto del
  estudio.
\item[$\square$]
  Explicar el enfoque experimental.
\item[$\square$]
  Explicar cómo los animales y modelos se usaron para explorar los
  objetivos científicos.
\item[$\square$]
  Si es relevante, explicar el impacto en la salud humana.
\item[$\square$]
  Comienza por explicar por qué es importante la investigación
\item[$\square$]
  Seguir por explicar el estado actual de la investigación
\item[$\square$]
  Seguir de hoyos o oproblemas en el campo.
\item[$\square$]
  Finalmente explicar cómo el estudio busca una solución a este
  problema.
\end{itemize}

(ChatGPT-10 2024; 2.0 2023; González and Likeador 2019; Hashtag 2020;
TerapiaExpress 2021; Postureo and HumildeFalsa 2022; EgoMeter and
BrainDrama 2023)

(PeerReviewHater and DataCherryPicker 2023; Butterfly 2024; Sarcasmo
2024; Falsacionista 2022)

(Peter Gould 2015-\/-2022; Carreyrou 2015)

\subsection{Objetivos}\label{objetivos}

\begin{enumerate}
\def\labelenumi{\roman{enumi}.}
\item
  Correlacionar puntuaciones de narcisismo encubierto con resistencia a
  la revisión por pares.
\item
  Cuantificar la frecuencia de frases pasivo-agresivas en correos de
  rechazo a colaboradores (``Lamento que no hayas entendido mi
  genialidad'').
\item
  Demostrar que su corteza prefrontal prioriza ``proteger mi ego'' sobre
  ``aceptar evidencia'' mediante fMRI durante discusiones académicas.
\item
  Desarrollar un algoritmo que detecte narcisismo encubierto en reviews
  anónimos (pista: busca la palabra ``claramente'' seguida de un insulto
  elegante).
\item
  Medir la actividad de la dopamina cuando alguien les dice ``Wow, eres
  tan auténtico''.
\end{enumerate}

\subsection{Hipótesis}\label{hipuxf3tesis}

Proponemos que el narcisista encubierto:

\begin{itemize}
\tightlist
\item
  Distorsiona el método científico para validar su autoimagen,
  convirtiendo hipótesis en horóscopos académicos (``Los datos no
  coinciden, pero mi intuición es un don'').
\item
  Usa la sección de conflicto de intereses para listar enemigos.
\item
  Su actividad cerebral al recibir críticas se asemeja a la de alguien
  viendo arder su trofeo de kindergarten.
\end{itemize}

\subsection{Metodolodía}\label{metodoloduxeda}

\emph{Metodología}: Este término es más amplio y no solo describe los
procedimientos específicos utilizados en el experimento, sino también
las bases teóricas que justifican esos métodos. Incluye una discusión
sobre por qué ciertos métodos son apropiados para la investigación en
cuestión. Si tu sección describe tanto el ``cómo'' (los pasos y
procedimientos) como el ``por qué'' (la justificación de la elección de
esos métodos).

\emph{Diseño Experimental}: Este término se refiere específicamente al
plan estructural de la investigación, cómo se organizan los
experimentos, qué variables se controlan, cómo se asignan los sujetos a
diferentes grupos, etc. Es más específico que ``Método'' y se centra en
la planificación del experimento en sí.

\emph{Método}: Es un término más específico y directo. Se enfoca en los
pasos concretos y técnicas empleadas en la investigación, sin
necesariamente entrar en detalles sobre la justificación teórica de esos
métodos.

\begin{itemize}
\tightlist
\item[$\square$]
  Escribir en pasado, con voz pasiva (\emph{Daniel cocinó una tortilla},
  \emph{Una tortilla fue cocinada por Danie}, \emph{Daniel la cocinó},
  \emph{América fue colonizada en 1492}, \emph{Se reparan automóviles
  /Se espera la renuncia del mandatario}), en contraste de voz activa
  (\emph{El presidente pronunció un largo discurso}, \emph{Varios
  millones visitan Barcelona cada año})
\end{itemize}

\begin{longtable}[]{@{}
  >{\raggedright\arraybackslash}p{(\columnwidth - 2\tabcolsep) * \real{0.4516}}
  >{\raggedright\arraybackslash}p{(\columnwidth - 2\tabcolsep) * \real{0.5484}}@{}}
\toprule\noalign{}
\begin{minipage}[b]{\linewidth}\raggedright
\textbf{Voz Activa}
\end{minipage} & \begin{minipage}[b]{\linewidth}\raggedright
\textbf{Voz Pasiva}
\end{minipage} \\
\midrule\noalign{}
\endhead
\bottomrule\noalign{}
\endlastfoot
El presidente pronunció un largo discurso & Un largo discurso fue
pronunciado por el presidente \\
Varios millones visitan Barcelona cada año & Barcelona es visitada cada
año por varios millones \\
Mi madre horneó una tarta de chocolate & Una tarta de chocolate fue
horneada por mi madre \\
Unos ladrones atracaron el banco & El banco fue atracado por unos
ladrone \\
\end{longtable}

\begin{longtable}[]{@{}ll@{}}
\toprule\noalign{}
\textbf{Verbo Activo} & \textbf{Verbo Pasivo} \\
\midrule\noalign{}
\endhead
\bottomrule\noalign{}
\endlastfoot
Escribe & Es escrito \\
Escribió & Fue escrito \\
Escribirá & Será escrito \\
Escriba & Sea escrito \\
Han escrito & Han sido escritos \\
\end{longtable}

\subsubsection{Animales}\label{animales}

\begin{itemize}
\tightlist
\item[$\square$]
  cuidado y monitoreo
\item[$\square$]
  Aprovación de comité de ética
\item[$\square$]
  Intervenciones y pasos utilizados para reducir dolor, sufrimiento y
  distrés.
\item[$\square$]
  Cómo se obtuvo el tamaño de la muestra a priori.
\end{itemize}

\subsection{Resultados y discusión}\label{resultados-y-discusiuxf3n}

\subsubsection{Scores de Narcisismo Encubierto correlacionan con rechazo
a
críticas}\label{scores-de-narcisismo-encubierto-correlacionan-con-rechazo-a-cruxedticas}

A mayor narcisismo encubierto, menor citación de colegas en papers
\texttt{(r\ =\ -0.99,\ p\ \textless{}\ 0.001)}.

Nivel de narcisismo vs.~Número de veces que dice `La literatura está
equivocada'.

\subsubsection{Activación cerebral en
fMRI}\label{activaciuxf3n-cerebral-en-fmri}

La amígdala se activa un 250\% más al escuchar ``Tu muestra es muy
pequeña'' vs.~``Tu teoría revolucionó la ciencia''.

\subsubsection{Análisis de texto}\label{anuxe1lisis-de-texto}

El 90\% de sus emails incluyen ``Con todo respeto\ldots{}'' seguido de
un ``Pero esto es basura'' disfrazado de citas de Popper.

\subsubsection{Nota}\label{nota}

Los sujetos negaron todos los resultados, diciendo ``Yo solo estoy aquí
para ayudar a la ciencia'' (clásico).

\subsection{Conclusión}\label{conclusiuxf3n}

El narcisismo encubierto es el virus silencioso de la mala ciencia:
convierte la duda en herejía, la colaboración en competencia, y los
journals en diarios íntimos con DOI. La solución no es expulsarlos, sino
mandarlos a un retiro espiritual con Carl Sagan de fondo (``El universo
no gira en torno a ti, Karen''). Propuesta: incluir tests de narcisismo
en las convocatorias de financiamiento. Cof cof.

Confirmamos que el narcisismo encubierto es como el ajo en las recetas:
todos creen que no lo usan, pero se huele a kilómetros. La neurociencia
sugiere que sus cerebros son máquinas de autoengaño sofisticadas, pero
con suficiente humor y memes, quizá podamos salvarlos (o al menos
reírnos en el proceso). Propuesta final: un bot de Twitter que responda
``Ok, Sigmund Freud'' a sus hilos existenciales.

\phantomsection\label{refs}
\begin{CSLReferences}{1}{0}
\bibitem[\citeproctext]{ref-NeurosisEnPijama2023}
2.0, Freud. 2023. \emph{Neurosis En Pijama: La Neurociencia Del Que Sube
Historias Motivacionales a Las 3 AM}. Silicon Valley: NeuroVanity Press.

\bibitem[\citeproctext]{ref-CienciaToxica2024}
Butterfly, Dr. I. Ron. 2024. \emph{Ciencia Tóxica: Cuando El Ego Es El
Principal {COI} (Conflicto de Intereses)}. Cambridge: Black Hole
Academic Press.

\bibitem[\citeproctext]{ref-HolmesFraude2021}
Carreyrou, John. 2015. {``Bad Blood: Secrets and Lies in a Silicon
Valley Startup.''} \emph{The Wall Street Journal}.

\bibitem[\citeproctext]{ref-IAyNarcisismo2024}
ChatGPT-10. 2024. {``¿Puede La {IA} Diagnosticar a Tu Ex? Un Análisis de
Mensajes Pasivo-Agresivos En Tinder.''} Reporte Técnico NTD-007.
Instituto de Tecnologías Dramáticas.

\bibitem[\citeproctext]{ref-FreudRecargado2023}
EgoMeter, Vanessa, and Ludwig BrainDrama. 2023. {``Why Do i Love Me? A
Neural Survey of Self-Admiration.''} \emph{Journal of Questionable
Personality Traits} 42 (6): 1337--50.
\url{https://doi.org/10.1234/jqpt.2023.007}.

\bibitem[\citeproctext]{ref-PopperLloruxf3n2022}
Falsacionista, Karl. 2022. {``El Narcisista y La Falsabilidad: Por Qué
Su Hipótesis Es Inmune a Los Hechos.''} In \emph{Psicopatologías de La
Razón: Filosofía Para Científicos Con Daddy Issues}, 777--89. Ediciones
Crítica Constructiva (Ironicamente).

\bibitem[\citeproctext]{ref-DopaminaYSelfies2019}
González, Dopamina, and Sergio Likeador. 2019. {``La Curva de La
Dopamina En Publicaciones de Café Artesanal: Un Estudio Con Narcisistas
Encubiertos.''} \emph{Annual Review of Pretentious Behavior} 7: 42--666.
\url{https://doi.org/10.6666/ar.pb.2019.007}.

\bibitem[\citeproctext]{ref-NeuroFilter2020}
Hashtag, Alan. 2020. {``NeuroFilter: Cómo Detectar Un {{`Humblebrag'}}
En 280 Caracteres o Menos.''} In \emph{Memes y Neuronas: La Ciencia
Detrás de Tu Timeline Tóxico}, edited by Twitter Académico, 69--420.
TikTok Academic Publications.

\bibitem[\citeproctext]{ref-EgoLab2023}
PeerReviewHater, Richard, and Ella DataCherryPicker. 2023. {``Narcissus
in the Lab: How Covert Narcissism Corrupts Citation Networks.''}
\emph{Journal of Irreproducible Results} 12: 45--69.
\url{https://doi.org/10.6666/jir.2023.045}.

\bibitem[\citeproctext]{ref-McGillCaseStudy}
Peter Gould, Vince Gilligan y. 2015-\/-2022. {``Better Call Saul:
Análisis de Chuck McGill Como Arquetipo Del Narcisista Encubierto En
Entornos de Alta Exigencia.''}

\bibitem[\citeproctext]{ref-SelfieCerebral2022}
Postureo, Carlos, and María HumildeFalsa. 2022. \emph{Humblebrags \&
Brain Scans: {fMRI} No Miente, Tu Perfil de Instagram Sí}. 1st ed.
Colección Para Narcisistas En Recuperación. Berlin: Springer.

\bibitem[\citeproctext]{ref-AlgorithmShade2024}
Sarcasmo, Ada. 2024. {``Detecting Academic Narcissism via
Passive-Aggressive LaTeX Comments: A Machine Learning Approach.''}
Technical Report IDPR-666. Institute of Dramatic Peer Review.
\url{https://arxiv.org/abs/2404.01joke}.

\bibitem[\citeproctext]{ref-DejaDeLeer2021}
TerapiaExpress, Ana. 2021. \emph{Deja de Leer Esto y Ve a Terapia: Guía
de Supervivencia Para El Autoengañado Crónico}. Buenos Aires: Ediciones
Ironía Sana.

\end{CSLReferences}



\end{document}
